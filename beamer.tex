% In your .tex file
% !TEX program = xelatex
\documentclass{beamer}
\usetheme{Copenhagen}
%\usepackage[ansinew]{inputenc}
\usepackage[latin1]{inputenc}
%\usepackage[applemac]{inputenc}

\title{Futures, Scheduling und Work Distribution}
\author{Christian Deckert}
\date{\today}
\institute{Universit\"at Mannheim}




\begin{document}

%
% Title
%
\maketitle

%
% TOC
%
\begin{frame}{Agenda}
\tableofcontents[currentsection]
\end{frame}


%
% Einführung
%
%
\section{Einf\"uhrung}

\subsection{Problemstellung}
%
% Problemstellung
%
\begin{frame}{Effektive Implementierung von parallelisierbaren Programmen}{Problemstellung}

\begin{itemize}
\item Aufteilung von Problemen in parallelisierbare Komponenten
\item Effektive Aufteilung der Arbeit
\item Verwaltung von Threads
\end{itemize}
\end{frame}


\subsection{Abgrenzung}
\begin{frame}{Multithreadingstrategien}

Multithreadingstrategien
\begin{columns}
    \column{.5\textwidth}
        \begin{block}{Peer-to-Peer}
        \begin{itemize}
            \item kurzlebige Threads
            \item 1 Aufgabe = 1 Thread
            \item Threads starten weitere (Sub-)Threads
        \end{itemize}
        \end{block}
    \column{.5\textwidth}
        \begin{block}{Delegation}
        \begin{itemize}
            \item Langlebige Arbeiter-Threads
            \item Aufgaben-Pools
            \item Bearbeitung des Aufgaben-Pools durch Arbeiter-Threads
        \end{itemize}
        \end{block}
    
\end{columns}

\begin{columns}
    \column{.5\textwidth}
        \begin{block}{Producer-Consumer}
        \begin{itemize}
            \item Producer generiert Daten
            \item Consumer verarbeitet Daten
        \end{itemize}
        \end{block}
    \column{.5\textwidth}
        \begin{block}{Pipeline}
        \begin{itemize}
            \item Fliessbandfertigung
            \item Weitergabe der Aufgabe von Thread zu Thread
        \end{itemize}
        \end{block}
    
\end{columns}
%http://books.google.de/books?id=k-gyap-yz7EC&pg=PT270&lpg=PT270&dq=delegation+(boss–worker)+example&source=bl&ots=bzZWsbxG4J&sig=lXaPvcv8Na5fWRiVMszcw8jI0g8&hl=de&sa=X&ei=fCRIVPzaHIfhywONnYKICw&ved=0CC0Q6AEwAQ#v=onepage&q=delegation%20(boss–worker)%20example&f=false
\end{frame}

\begin{frame}{Vorteile von Delegation \"uber Peer-to-Peer}

\begin{columns}
    \column{.5\textwidth}
        Peer-to-Peer Nachteile
        
        \begin{itemize}
        \item Kosten f\"ur Memory, Setup, Teardown der Threads
        \item Hoher Overhead pro Komponente
        \item kurzlebige Threads
        \end{itemize}
    \column{.5\textwidth}
        Delegations Vorteile
        \begin{itemize}
        \item Geringer Overhead pro Komponente
        \item Trennung von Aufgabe und Ausf\"uhrung
        \item Langlebige Threads
        \end{itemize}
    
\end{columns}
\end{frame}

\subsection{Begriffskl\"ahrung: Futures}

\begin{frame}{Abh\"angigkeiten zwischen Aufgaben}
\begin{itemize}
\item Aufgaben sind nicht (immer) unabh\"angig
\item andere Aufgaben m\"ussen abgeschlossen werden
\item noch unbekannte Ergebnisse werden als Futures bezeichnet
\end{itemize}
\end{frame}





\end{document}